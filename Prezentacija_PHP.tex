\documentclass{beamer}
%
% Choose how your presentation looks.
%
% For more themes, color themes and font themes, see:
% http://deic.uab.es/~iblanes/beamer_gallery/index_by_theme.html
%
\mode<presentation>
{
  \usetheme{Warsaw}      % or try Darmstadt, Madrid, Warsaw, ...
  \usecolortheme{crane} % or try albatross, beaver, crane, ...
  \usefonttheme{default}  % or try serif, structurebold, ...
  \setbeamertemplate{navigation symbols}{}
  \setbeamertemplate{caption}[numbered]
} 

\usepackage[english,serbian]{babel}
\usepackage[utf8]{inputenc}
\usepackage[T2A]{fontenc} % enable Cyrillic fonts
\usepackage{currvita}

\title[PHP]{Programski jezik PHP}%short, full
\author[Vučković, Ivanović, Simić, Stevović]{Đorđe Vučković, Tamara Ivanović,\\ Petar Simić, Stefan Stevović}
\institute{\small{Seminarski rad u okviru kursa\\Metodologija stručnog i naučnog rada\\ Matematički fakultet}}
\date{Maj, 2019.}

\begin{document}

\begin{frame}
  \titlepage
\end{frame}

% Uncomment these lines for an automatically generated outline.
%\begin{frame}{Outline}
%  \tableofcontents
%\end{frame}

%%%%%%%%%%%%%%%%%%%%%%%%%%%%%%%%%%
\section{Uvod}

\begin{frame}{Uvod}

\begin{itemize}
  \item Hypertext preprocessor
  \item Skriptni jezik opšte namene
  \item Nestriktna semantika
  \item U prvih 6 jezika na GitHub-u
  \item Veliki broj radnih okuženja
  \item ,,Learning PHP", ,,PHP Cookbook", ,,Programming PHP"
  
\end{itemize}

\end{frame}

%%%%%%%%%%%%%%%%%%%%%%%%%%%%%%%%%%
\section{Programski jezik PHP}
\subsection{Istorijski razvoj}
\begin{frame}{Istorijski razvoj}
	\begin{itemize}
		\item Rasmus Lerdorf, 1994. godina
		\item Niz C skriptova - PHP Tools
		\item FI (eng. Forms Interpreter)
		\item 1996. - Alat FI evoluirao u programski jezik PHP/FI
		\item Zeev Suraski i Andi Gutmans - redizajnirano PHP/FI jezgro, novi programski jezik Hypertext Preprocessor
		\item Zend Engine - modularnost, leksička analiza, upravljanje memorijom...
		\item PHP 4.0 - poboljšane performanse, podrška HTTP sesijama, novi jezički konstrukti
	
	
	\end{itemize}
    
\end{frame}

%%%%%%%%%%%%%%%%%%%%%%%%%%%%%%%%%%
\subsection{Mogućnosti PHP-a}
\begin{frame}{Mogućnosti PHP-a}
    
\end{frame}

%%%%%%%%%%%%%%%%%%%%%%%%%%%%%%%%%%%%
\subsection{Implementacije PHP-a}
\begin{frame}{Implementacije PHP-a}
    
\end{frame}

\subsection{Podržane paradigme}
\begin{frame}{Podržane paradigme}
    
\end{frame}


%%%%%%%%%%%%%%%%%%%%%%%%%%%%%%%%%%%%%%
\subsection{Najpoznatija okruženja}
\begin{frame}{Najpoznatija okruženja}
    
\end{frame}

%%%%%%%%%%%%%%%%%%%%%%%%%%%%%%%%%%%%%%%
\subsection{Instalacija na Windows OS}

\begin{frame}{Instalacija na Windows OS}


% Commands to include a figure:
%\begin{figure}
%\includegraphics[width=\textwidth]{your-figure's-file-name}
%\caption{\label{fig:your-figure}Caption goes here.}
%\end{figure}

\end{frame}

\subsection{Instalacija na Linux OS}

\begin{frame}{Instalacija na Linux OS}

\end{frame}

%%%%%%%%%%%%%%%%%%%%%%%%%%%%%%%%%%%%%%%%
\subsection{Primer k\^{o}da}
\begin{frame}{Primer k\^{o}da}
    
\end{frame}

%%%%%%%%%%%%%%%%%%%%%%%%%%%%%%%%%%%%%%%
\subsection{Specifičnosti jezika}
\begin{frame}{Specifičnosti jezika}
    
\end{frame}

%%%%%%%%%%%%%%%%%%%%%%%%%%%%%%%%%%%%%%%%%
\section{Zaključak}
\begin{frame}{Zaključak}
    
\end{frame}

%%%%%%%%%%%%%%%%%%%%%%%%%%%%%%%%%%%%%%%%%
\section{Literatura}
\begin{frame}{Literatura}
    \begin{itemize}
        \item L. Atkinson. Core PHP Programming. Prentice Hall, 2003.
        \item https://www.php.net/manual/
        \item R. Lerdorf and K. Tatroe. Programming PHP. O'Reilly Media, 2002
        \item A. Trachtenberg and D. Sklar. PHP Cookbook, 3rd Edition,  O'Reilly Media, 2014
    \end{itemize}
\end{frame}

\end{document}
